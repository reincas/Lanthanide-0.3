\documentclass[dvips,landscape,a4paper]{slides}
%\usepackage{t1enc}
\usepackage[latin1]{inputenc}
\usepackage{epsfig}
\usepackage{amsmath}
%\usepackage{amstex}
\usepackage[german]{babel}

\addtolength{\topmargin}{-30mm}
\addtolength{\textheight}{45mm}

\def\einheit#1#2{\ensuremath{\mathrm{#1\,#2}}}
\def\chem#1{\ensuremath{\mathrm{#1}}}
\def\stacked#1#2{\stackrel{\scriptstyle #1}{\scriptstyle #2}}

\def\shalf{{\scriptstyle\frac12}}

\def\vector#1{\vec{#1}}
\def\tensor#1#2{\hat{#1}^{{(#2)}}}
\def\scalartensor#1{\hat{#1}}
\def\element#1#2#3{#1_{#2}^{{(#3)}}}

\def\textvector#1{\vec{\text{#1}}}
\def\texttensor#1#2{\hat{\text{#1}}^{\text{{(#2)}}}}
\def\textscalartensor#1{\hat{\text{#1}}}
\def\textelement#1#2#3{\text{#1}_{\text{#2}}^{\text{{(#3)}}}}

\def\dot#1#2{({#1} \cdot {#2})}
\def\threedot#1#2#3{({#1} \cdot {#2} \cdot {#3})}
\def\bigdot#1#2{\big({#1} \cdot {#2}\big)}
\def\cross#1#2#3{\{{#1} \!\times {#2}\}^{(#3)}}
\def\crosstop#1#2#3#4{\{{#1} \!\times {#2}\}^{(#3){#4}}}
\def\bigcross#1#2#3{\big\{{#1} \!\times {#2}\big\}^{(#3)}}
\def\elcross#1#2#3#4{\{{#1} \!\times {#2}\}_{#3}^{(#4)}}
\def\bigelcross#1#2#3#4{\big\{{#1} \!\times {#2}\big\}_{#3}^{(#4)}}

\def\la{l_a}\def\mla{m_{l_a}}\def\msa{m_{s_a}}
\def\lb{l_b}\def\mlb{m_{l_b}}\def\msb{m_{s_b}}
\def\lc{l_c}\def\mlc{m_{l_c}}\def\msc{m_{s_c}}
\def\ld{l_d}\def\mld{m_{l_d}}\def\msd{m_{s_d}}

\def\su{\rule{0ex}{2.7ex}}
\def\sd{\rule[-1.7ex]{0ex}{1.7ex}}

\def\iiij#1#2#3#4#5#6{%
  \begin{pmatrix}#1&#2&#3\\#4&#5&#6\end{pmatrix}}
\def\vij#1#2#3#4#5#6{%
  \left\{\begin{matrix}#1&#2&#3\\#4&#5&#6\end{matrix}\right\}}

\def\head#1{
  \begin{center}
    \bfseries #1\relax
  \end{center}
  }

\def\source#1{
  \begin{flushright}
    \tiny #1\relax
  \end{flushright}
  }

\begin{document}

%%%%%%%%%%%%%%%%%%%%%%%%%%%%%%%%%%%%%%%%%%%%%%%%%%%%%%%%%%%%%%%%%%%%%%%%%%%%%

\begin{slide}
  \begin{center}
    \Large\bfseries
    Hamilton Operators \\
    of $\text{l}^\text{N}$ Configurations
  \end{center}
  \vspace{3\baselineskip}
  \begin{center}
    Reinhard Caspary
  \end{center}
\end{slide}

%%%%%%%%%%%%%%%%%%%%%%%%%%%%%%%%%%%%%%%%%%%%%%%%%%%%%%%%%%%%%%%%%%%%%%%%%%%%%

\begin{slide}
  \vspace*{\fill}
  \head{Central field approximation}

  Hamiltonian of an ion with a nucleus of charge $Z$ and $N$ electrons
  in partly filled shells:
  \begin{displaymath}
    H = -\sum_{i}\frac{p_i^2}{2m_e}
    -\sum_{i}\frac{Ze^2}{r_i}
    + \sum_{i<j}\frac{e^2}{r_{ij}}
  \end{displaymath}

  Approximation: Every electron is located in a central potential $U$,
  caused by the coulomb energy of the nucleus and the other electrons:
  \begin{displaymath}
    H_0 = \sum_{i} \left[-\frac{p_i^2}{2m_e}+U(\vector{r}_i)\right]
  \end{displaymath}

  $H_0$ therefore is a sum of one-electron operators.

  \vspace*{\fill}
\end{slide}

%%%%%%%%%%%%%%%%%%%%%%%%%%%%%%%%%%%%%%%%%%%%%%%%%%%%%%%%%%%%%%%%%%%%%%%%%%%%%

\begin{slide}
  \vspace*{\fill}
  \head{Eigenfunctions of $\text{H}_\text{0}$}
  
  The solution of the Schr�dinger equation of $H_0$ for one electron
  splits into a radial and an orbital part as for the Hydrogen atom:
  \begin{displaymath}
    \psi_{\alpha} = \frac1r
    R_{nl}(r)Y^l_{m_l}(\theta,\phi)\chi_{m_s}
  \end{displaymath}
  
  where $\alpha = (nlm_lm_s)$. In the case of $N$ electrons, the
  radial-orbit determinant of these eigenfunctions provides an
  antisymetric solution of $H_0$, the so-called determinantal product
  states:
  \begin{displaymath}
    \Psi_0 = 
    \frac{\det[\psi_{\alpha_j}(i)]}{\sqrt{N}} =
    \left\{\alpha_1\alpha_2\ldots\alpha_N\right\}
  \end{displaymath}
  
  Inside a given configuration, all states $\Psi_0$ are completely
  degenerated with the energy $E_0$.

  \vspace*{\fill}
\end{slide}

%%%%%%%%%%%%%%%%%%%%%%%%%%%%%%%%%%%%%%%%%%%%%%%%%%%%%%%%%%%%%%%%%%%%%%%%%%%%%

\begin{slide}
  \vspace*{\fill}
  \head{Perturbation theory}
  
  All interactions, which are not contained in $H_0$ are sumed up in
  the perturbation operator
  \begin{displaymath}
    E = E_0 + \sum_i\langle\Psi'_0|H_i|\Psi_0\rangle
  \end{displaymath}

  The radial parts of the perturbation matrix elements are treated as
  parameters $X^i$, as they depend on the unknown Potential~$U$. The
  orbital part $x_i$ of the matrix elements is calculated:
  \begin{displaymath}
    E = E_0 + \sum_i X^i x_i
  \end{displaymath}

  \vspace*{\fill}
\end{slide}

%%%%%%%%%%%%%%%%%%%%%%%%%%%%%%%%%%%%%%%%%%%%%%%%%%%%%%%%%%%%%%%%%%%%%%%%%%%%%

\begin{slide}
  \vspace*{\fill}
  \head{Free ion hamiltonian}
  
  The free ion hamiltonian consists of operators of electric and magnetic
  interactions inside the ground configuration and of effective
  operators that take interactions with other configurations into
  account.
  
  For the $\text{f}^N$ configuration, the following radial parameters
  and the appropriate angular operators apear:

  \begin{center}
    \begin{tabular}{|c|l|l|}
      \hline
      \su& \multicolumn{1}{c|}{Intra-configuration}
      \sd& \multicolumn{1}{c|}{Inter-configuration} \\
      \hline
      electric & 
      \begin{tabular}[c]{rl}
        $H_1$: & $F^2,F^4,F^6$ \\
      \end{tabular} &
      \begin{tabular}[c]{rl}
        \su$H_3$: & $\alpha,\beta,\gamma$ \\
        \sd$H_4$: & $T^2,T^3,T^4,T^6,T^7,T^8$
      \end{tabular} \\
      \hline
      magnetic &
      \begin{tabular}[c]{rl}
        \su$H_2$: & $\zeta$ \\
        \sd$H_5$: & $M^0,M^2,M^4$
      \end{tabular} &
      \begin{tabular}[c]{rl}
        $H_6$: & $P^2,P^4,P^6$ \\
      \end{tabular} \\
      \hline
    \end{tabular}
  \end{center}

  \vspace*{\fill}
\end{slide}

%%%%%%%%%%%%%%%%%%%%%%%%%%%%%%%%%%%%%%%%%%%%%%%%%%%%%%%%%%%%%%%%%%%%%%%%%%%%%

\begin{slide}
  \vspace*{\fill}
  \head{Coulomb interaction}
  
  The first component of the perturbation operator is the remaining
  part of $H$ not contained in $H_0$:
  \begin{displaymath}
    H-H_0 = \sum_{i=1}^{N}\left[-\frac{Ze^2}{r_i}-U(\vector{r}_i)\right]
    + \sum_{i<j}\frac{e^2}{r_{ij}}
  \end{displaymath}

  For all states of a given configuration the first part leads to the
  same energy shift, therefore
  \begin{displaymath}
    H_1
    = \sum_{i<j}\frac{e^2}{r_{ij}}
    = e^2\, \sum_{i<j}\, \frac{r_<^k}{r_>^{k+1}}\,
    \big(\tensor{c}{k}_i\cdot\tensor{c}{k}_j\big)
  \end{displaymath}

  The energy shift is
  \begin{displaymath}
    E_1 = \sum_{\stacked{k=2}{k\,\text{even}}}^{2l} F^k f_k
    \qquad
    \scalartensor{f}_k =
    \sum_{i<j}
    \big(\tensor{c}{k}_i\cdot\tensor{c}{k}_j\big)
  \end{displaymath}
  
  \vspace*{\fill}
\end{slide}

%%%%%%%%%%%%%%%%%%%%%%%%%%%%%%%%%%%%%%%%%%%%%%%%%%%%%%%%%%%%%%%%%%%%%%%%%%%%%

\begin{slide}
  \vspace*{\fill}
  \head{Spin-orbit interaction}
  
  The hamiltonian of the spin-orbit interaction is given by the
  following expression:
  \begin{displaymath}
    H_2 = \sum_i
    \frac{\hbar^2}{2m^2c^2r_i}\,
    \frac{\text{d}U(\vector{r}_i)\,}{\text{d}r_i}\,
    \dot{\vector{l}_i}{\vector{s}_i}
  \end{displaymath}

  The energy shift is
  \begin{align*}
    E_2 &= \zeta_{nl}\,Z_{so}
    \\[.667\baselineskip]
    \scalartensor{Z}_{so} &=
    \sum_i
    \dot{\vector{l}_i}{\vector{s}_i}
  \end{align*}

  \vspace*{\fill}
\end{slide}

%%%%%%%%%%%%%%%%%%%%%%%%%%%%%%%%%%%%%%%%%%%%%%%%%%%%%%%%%%%%%%%%%%%%%%%%%%%%%

\begin{slide}
  \vspace*{\fill}
  \head{Coulomb configuration interaction I}
  
  The coulomb interaction of the $l^N$ configuration with the
  three types of configurations
  \begin{displaymath}
    l^{n-2}l'^2 \,/\, l^{n-2}l'l'' \,,
    \quad
    l'^{4l'}l^{n+2} \,/\, l'^{4l'+1}l''^{4l''+1}l^{n+2} \,\text{, and}
    \quad
    l'^{4l'+1}l^{n}l''
  \end{displaymath}
  leads to the following energy shift:
  \vspace{-.5\baselineskip}
  \begin{quote}
    \begin{itemize}
    \item $l = 1$:\qquad
      $E_3 = \alpha L(L+1)$
    \item $l = 2$:\qquad
      $E_3 = \alpha L(L+1) + \beta G(R_5)$
    \item $l = 3$:\qquad
      $E_3 = \alpha L(L+1) + \beta G(G_2) + \gamma G(R_7)$
    \end{itemize}
  \end{quote}
  \vspace{-.5\baselineskip}
  
  $L(L+1)$ is the matrix element of $\vector{L}^2$, $G(S)$ is the
  matrix element of Casimir's operator of the symetry group $S$.

  \vspace*{\fill}
\end{slide}

%%%%%%%%%%%%%%%%%%%%%%%%%%%%%%%%%%%%%%%%%%%%%%%%%%%%%%%%%%%%%%%%%%%%%%%%%%%%%

\begin{slide}
  \vspace*{\fill}
  \head{Coulomb configuration interaction II}
  
  The coulomb interaction of the $l^N$ configuration with the
  remaining two relevant types of configurations
  \begin{displaymath}
    l^{n-1}l'
    \quad \text{and} \quad
    l'^{4l'+1}l^{n+1}
  \end{displaymath}
  leads to the following energy shift:
  \begin{displaymath}
    E_4 = \sum_{n} T^n t_n
  \end{displaymath}
  \begin{displaymath}
    \scalartensor{t}_{n} =
    3!\,
    \langle k k' k'' | n \rangle\,
    \sum_{i<j<k}
    \threedot{\tensor{v}{k}_i}{\tensor{v}{k'}_j}{\tensor{v}{k''}_k}
  \end{displaymath}
  
  The parameters $k$, $k'$, and $k''$ may have even values between 2
  and $2l$ and their order is of no importance. Therefore $n$ may run
  from 1 to $l^2+l!/[(l-3)!3!]$. But in case of $\text{f}^N$
  configurations, only the $T^n$'s with $n=2,3,4,6,7,8$ are lineary
  independant.

  \vspace*{\fill}
\end{slide}

%%%%%%%%%%%%%%%%%%%%%%%%%%%%%%%%%%%%%%%%%%%%%%%%%%%%%%%%%%%%%%%%%%%%%%%%%%%%%

\begin{slide}
  \vspace*{\fill}
  \head{Spin-spin and spin-other-orbit interaction}
  
  \vspace{\baselineskip}
  The hamiltonian of the spin-spin and spin-other-orbit interaction is
  given by the expression
  \vspace{0.333\baselineskip}
  \begin{displaymath}
    H_5 = \frac{e^2\hbar^2}{m^2c^2} \sum_{i<j}
    \bigg[
    \frac{\dot{\vector{s}_i\!}{\!\vector{s}_i}}{\vector{r}_{ij}^{\,3}} -
    \frac{3 \dot{\vector{r}_{ij}\!}{\!\vector{s}_i}
      \dot{\vector{r}_{ij}\!}{\!\vector{s}_j}}{r_{ij}^5}
    +
    \bigdot{[\vector{\nabla}_{\!i}\frac{1}{r_{ij}} \!\times\! \vector{p}_i]}%
    {[\vector{s}_i \!+\! 2\vector{s}_j]}
    \bigg]
  \end{displaymath}

  \vspace{\baselineskip}
  The energy shift is
  \begin{displaymath}
    E_5 = \sum_{\stacked{k=0}{k\,\text{even}}}^{2l-2} M^k m_k
  \end{displaymath}

  \vspace*{\fill}
\end{slide}

%%%%%%%%%%%%%%%%%%%%%%%%%%%%%%%%%%%%%%%%%%%%%%%%%%%%%%%%%%%%%%%%%%%%%%%%%%%%%

\begin{slide}
  \vspace*{\fill}
  \head{Operator $\textscalartensor{m}_\text{k}$}
  
  \begin{align*}
    \scalartensor{m}_k =&
    \sum_{i\neq j}
    \bigg[a\,
    \big[
    \crosstop{\tensor{w}{0,k}_i}{\tensor{w}{1,k+1}_j}{11}{0} +
    2\,\crosstop{\tensor{w}{0,k+1}_i}{\tensor{w}{1,k}_j}{11}{0}
    \big]
    \\[0\baselineskip]
    &+b\,
    \big[
    \crosstop{\tensor{w}{0,k+2}_i}{\tensor{w}{1,k+1}_j}{11}{0} +
    2\,\crosstop{\tensor{w}{0,k+1}_i}{\tensor{w}{1,k+2}_j}{11}{0}
    \big]
    \\[0\baselineskip]
    &+c\,
    \crosstop{\tensor{w}{1,k}_i}{\tensor{w}{1,k+2}_j}{22}{0}
    \bigg]
  \end{align*}
  \source{Judd 1968: (1), (2)}

  \vspace{-0.667\baselineskip}
  \begin{align*}
    a &=
    \langle l || \tensor{C}{k} || l \rangle^2\,
    \sqrt{(k+1)(2l+k+2)(2l-k)}
    \\[0.333\baselineskip]
    b &=
    \langle l || \tensor{C}{k+2} || l \rangle^2\,
    \sqrt{(k+2)(2l+k+3)(2l-k-1)}
    \\[0.333\baselineskip]
    c &=
    -2
    \langle l || \tensor{C}{k} || l \rangle\,
    \langle l || \tensor{C}{k+2} || l \rangle\,
    \sqrt{(k+1)(k+2)(2k+3)}
  \end{align*}

  \vspace*{\fill}
\end{slide}

%%%%%%%%%%%%%%%%%%%%%%%%%%%%%%%%%%%%%%%%%%%%%%%%%%%%%%%%%%%%%%%%%%%%%%%%%%%%%

\begin{slide}
  \vspace*{\fill}
  \head{Spin-orbit configuration interaction}
  
  The spin-orbit interaction of the $(nl)^N$ configuration with the three
  configurations
  \begin{displaymath}
    (nl)^{N-1} n'l \,,
    \quad
    (n'l)^{4l+1} (nl)^{N+1} \,\text{, and}
    \quad
    (n'l')^{4l'+1} (nl)^N n''l'
  \end{displaymath}
  leads to the energy shift

  \begin{displaymath}
    E_6 = \sum_{\stacked{k=2}{k\,\text{even}}}^{2l} P^k p_k
  \end{displaymath}

  \vspace*{\fill}
\end{slide}

%%%%%%%%%%%%%%%%%%%%%%%%%%%%%%%%%%%%%%%%%%%%%%%%%%%%%%%%%%%%%%%%%%%%%%%%%%%%%

\begin{slide}
  \vspace*{\fill}
  \head{Operator $\textscalartensor{p}_\text{k}$}
  
  \begin{multline*}
    \scalartensor{p}_k =
    \langle l || \tensor{C}{k} || l \rangle^2\,
    \frac{1}{\sqrt{2(2k+1)}}
    \\[0.5\baselineskip]
    \times
    \sum_{i\neq j}
    \bigg[
    \sqrt{\frac{(2l+k+2)(2l-k)(k+1)}{2k+3}}\ 
    \crosstop{\tensor{w}{0,k}_i}{\tensor{w}{1,k+1}_j}{11}{0}
    \\[0.5\baselineskip]
    -\sqrt{\frac{(2l+k+1)(2l-k+1)k}{2k+1}}\ 
    \crosstop{\tensor{w}{0,k}_i}{\tensor{w}{1,k-1}_j}{11}{0}
    \bigg]
  \end{multline*}
  \source{Goldschmidt 1978: (1.156)}

  \vspace*{\fill}
\end{slide}

%%%%%%%%%%%%%%%%%%%%%%%%%%%%%%%%%%%%%%%%%%%%%%%%%%%%%%%%%%%%%%%%%%%%%%%%%%%%%

\begin{slide}
  \vspace*{\fill}
  \head{Crystal field interaction}
  
  The coulomb interaction of the electrons with an external charge
  distribution $\varrho(\vector{r})$ is given by
  \begin{displaymath}
    H_7 =
    -e\,\varrho(\vector{r})
    \sum_i
    \sum_{k,q}
    \frac{r_i^k}{r^{k+1}}
    \sqrt{\frac{4\pi}{2k+1}}
    Y_q^k(\theta,\varphi)
    (\element{C}{q}{k})_i
  \end{displaymath}

  The energy shift is
  \begin{align*}
    E_7 &=
    \sum_{\stacked{k=2}{k\,\text{even}}}^{2l}
    \sum_{q\,\text{even}}
    B_q^k\, \sum_i (\element{C}{q}{k})_i
    \\[0.333\baselineskip]
    B_{-q}^k &= (-1)^q B_q^k
  \end{align*}

  \vspace*{\fill}
\end{slide}

%%%%%%%%%%%%%%%%%%%%%%%%%%%%%%%%%%%%%%%%%%%%%%%%%%%%%%%%%%%%%%%%%%%%%%%%%%%%%

\begin{slide}
  \vspace*{\fill}
  \head{Magnetic field interaction}
  
  The hamiltonian of an external magnetic field is
  \begin{displaymath}
    H_8 =
    \mu\, \bigdot{\vector{B}}{\sum_i[\vector{l}_i+g_s\vector{s}_i]}
  \end{displaymath}

  The energy shift is
  \begin{align*}
    E_8 &= B_z Z_m
    \\[0.333\baselineskip]
    \scalartensor{Z}_m &= \mu
    \sum_i[(l_z)_i+g_s(s_z)_i]
  \end{align*}

  \vspace*{\fill}
\end{slide}

%%%%%%%%%%%%%%%%%%%%%%%%%%%%%%%%%%%%%%%%%%%%%%%%%%%%%%%%%%%%%%%%%%%%%%%%%%%%%

\end{document}
